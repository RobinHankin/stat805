% -*- mode: noweb; noweb-default-code-mode: R-mode; -*-
\documentclass[nojss]{jss}\usepackage[]{graphicx}\usepackage[]{color}
% maxwidth is the original width if it is less than linewidth
% otherwise use linewidth (to make sure the graphics do not exceed the margin)
\makeatletter
\def\maxwidth{ %
  \ifdim\Gin@nat@width>\linewidth
    \linewidth
  \else
    \Gin@nat@width
  \fi
}
\makeatother

\definecolor{fgcolor}{rgb}{0.345, 0.345, 0.345}
\newcommand{\hlnum}[1]{\textcolor[rgb]{0.686,0.059,0.569}{#1}}%
\newcommand{\hlstr}[1]{\textcolor[rgb]{0.192,0.494,0.8}{#1}}%
\newcommand{\hlcom}[1]{\textcolor[rgb]{0.678,0.584,0.686}{\textit{#1}}}%
\newcommand{\hlopt}[1]{\textcolor[rgb]{0,0,0}{#1}}%
\newcommand{\hlstd}[1]{\textcolor[rgb]{0.345,0.345,0.345}{#1}}%
\newcommand{\hlkwa}[1]{\textcolor[rgb]{0.161,0.373,0.58}{\textbf{#1}}}%
\newcommand{\hlkwb}[1]{\textcolor[rgb]{0.69,0.353,0.396}{#1}}%
\newcommand{\hlkwc}[1]{\textcolor[rgb]{0.333,0.667,0.333}{#1}}%
\newcommand{\hlkwd}[1]{\textcolor[rgb]{0.737,0.353,0.396}{\textbf{#1}}}%
\let\hlipl\hlkwb

\usepackage{framed}
\makeatletter
\newenvironment{kframe}{%
 \def\at@end@of@kframe{}%
 \ifinner\ifhmode%
  \def\at@end@of@kframe{\end{minipage}}%
  \begin{minipage}{\columnwidth}%
 \fi\fi%
 \def\FrameCommand##1{\hskip\@totalleftmargin \hskip-\fboxsep
 \colorbox{shadecolor}{##1}\hskip-\fboxsep
     % There is no \\@totalrightmargin, so:
     \hskip-\linewidth \hskip-\@totalleftmargin \hskip\columnwidth}%
 \MakeFramed {\advance\hsize-\width
   \@totalleftmargin\z@ \linewidth\hsize
   \@setminipage}}%
 {\par\unskip\endMakeFramed%
 \at@end@of@kframe}
\makeatother

\definecolor{shadecolor}{rgb}{.97, .97, .97}
\definecolor{messagecolor}{rgb}{0, 0, 0}
\definecolor{warningcolor}{rgb}{1, 0, 1}
\definecolor{errorcolor}{rgb}{1, 0, 0}
\newenvironment{knitrout}{}{} % an empty environment to be redefined in TeX

\usepackage{alltt}
\usepackage{amsmath}
%%%%%%%%%%%%%%%%%%%%%%%%%%%%%%
%% declarations for jss.cls %%%%%%%%%%%%%%%%%%%%%%%%%%%%%%%%%%%%%%%%%%
%%%%%%%%%%%%%%%%%%%%%%%%%%%%%%

%% just as usual
\author{Auckland University of Technology}
\title{Assignment 1, STAT805}
%\VignetteIndexEntry{The multivator package}

%% for pretty printinng and a nice hypersummary also set:
\Plainauthor{Robin K. S. Hankin}
\Plaintitle{Math}
\Shorttitle{Math}


\newcommand{\question}{{\bf Question: }}
\newcommand{\answer}{{\bf Answer: }}
\newcommand{\Rq}{{\bf ``Is R doing what I think it is doing?''}}

%% an abstract and keywords

\Abstract{ All four questions have equal value: 10 marks for the
  question itself and 15 for the verification of your work.}

\Keywords{R, calculus}
\Plainkeywords{R}



\Address{
  Robin K. S. Hankin\\
  WT118\\
  Auckland University of Technology\\
  Auckland\\
  New Zealand\\
}


%% need no \usepackage{Sweave.sty}
\IfFileExists{upquote.sty}{\usepackage{upquote}}{}
\begin{document}  
\section{Calculus (univariate)}


\subsection*{Question 1}

Consider the  function defined by:

\[
f(x) = \sin(\sin(x^2)) + \cos\left(\frac{x}{1+x+2x^2}\right)-1
\]


\begin{itemize}
\item Translate this into R idiom and plot it for
  $x\in\left[-2,2\right]$.
  \item Use {\tt uniroot()} to find all four roots of $f(\cdot)$ in
    the interval $\left[-2,2\right]$.
\item Observe how difficult it would be to differentiate this function
  symbolically (not impossible!  Just a PITA).  Use R to plot an
  approximate derivative in the same range.
\item Recall that if $x_0$ is a local maximum, then
  $f'(x_0)=0$.  But observe how difficult it would be to solve
  $\frac{df(x)}{dx}=0$ for $x$ symbolically.  Now use R to numerically minimize the
  function in the interval $\left[-1,1\right]$, using {\tt optimize()}
  \item find both local maxima in the interval $\left[-2,2\right]$.
\item Find the area under the curve from 0 to 2 using {\tt integrate()}. 
\end{itemize}

\subsection*{Question 2}

Consider the function

\[  
f(x,y) = x^2+y^2+4y+ 7\sin(x+y^2)
\]

\begin{itemize}
\item Translate this function into R idiom and evaluate $f(0.4,0.5)$.
\item Plot a contour graph of this function in the range $-3\leq x,y\leq 3$.
\item Use {\tt optim()} to find the local minimum of this function near
$(-1,1)$.
\item Use {\tt cubature::hcubature()} to find the volume enclosed
between the x-y plane and the function surface in the plotting region.
\end{itemize}
    
\subsection*{Question 3}
 
Consider the matrix $M$ defined by


\[
\begin{bmatrix}
  5&1&1\\
  1&2&1\\
  1&1&2
  \end{bmatrix}
\]


\begin{itemize}
\item Translate this matrix into R and calculate $\operatorname{det}\left(M\right)$ and $\operatorname{tr}\left(M\right)$. 
\item Find $M^{-1}$ using {\tt solve()} and verify that this is correct.
\item Find the eigenvectors and eigenvalues of $M$ and demonstrate the validity of your result
\item Consider instead the matrix $N=N(x)=\begin{bmatrix}
  5&1&x\\
  1&2&1\\
  x&1&2
  \end{bmatrix}$, where $x$ is a real number.  Plot $\operatorname{det}\left(\operatorname{N}(x)\right)$  for a sensible range of $x$.  Say something interesting about the graph.  
  \item  (harder) Consider the matrix $P=P(x,y)=\begin{bmatrix}
  5&y&x\\
  y&2&1\\
  x&1&x
  \end{bmatrix}$, where $x,y$ are real numbers.  Say something interesting about the eigenvalues of $P$, considered as a function of $x$ and~$y$.
\end{itemize} 

\subsection*{Question 4}

\begin{itemize}
  \item 
    Bearing in mind that question 4 has equal course credit to the
    other questions, choose any single question above, and generalise
    the question in some way (perhaps you could consider a function
    with an additional parameter).  Credit will be given for
    appropriate reasoning, quantification of any numerical error, and
    verification of results.
    \end{itemize}


\bibliography{multivator} \end{document}

